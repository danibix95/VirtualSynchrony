\documentclass[11pt]{article}

\usepackage{geometry}
\geometry{
	a4paper,
	top=2cm
}
\usepackage[utf8x]{inputenc}
\usepackage{graphicx}
\usepackage{float}
\usepackage[hidelinks]{hyperref}
\usepackage{cleveref}
\usepackage{amsmath}
\usepackage{enumerate}
\usepackage{url}
\usepackage{algorithm}
\usepackage{algorithmic}
\usepackage{amsmath}

\renewcommand{\algorithmicforall}{\textbf{for each}}

\title{\LARGE{\textbf{Virtual Synchrony - Project Report}\\[0mm]\large{Distributed Systems 1, 2017 - 2018}}}
\author{Daniele Bissoli - 197810, Andrea Zampieri - 198762}
\date{}

\begin{document}
	\maketitle
	
	\section{Introduction}
	In this project, we have implemented a simple group communication service that permits different members of a group to communicate through messages exchange. The management of the group is performed in a centralized fashion through a special node, named \textit{Group Manager}. As design requirement, the system is able to guarantee the Virtual Synchrony, providing the synchronization between members.
	
	\subsection{View and Virtual Synchrony}
	A view is a local representation of the system, that is the set of group members that were still operational when last group update (join/crash) has been performed. Each member of the group has its own view and they have to globally agree on the same view in order to consistently communicate. Moreover, a view can be seen as a system epoch, since it changes every time a new participant join the group or when a cohort is detected to be crashed.
	Under the assumptions of reliable links and FIFO channels, \textit{Virtual Synchrony} is the mechanism which guarantees that messages sent in a specific epoch are delivered to all the group members and just those participant in that epoch.
	
	\section{Project Architecture}
	The project is divided into three main classes (\textit{Node}, \textit{Participant} and \textit{Group Manager}). In addition, it contains the definition of all exchanged messages and the class in charge of managing the creation of the system. In the following paragraphs the three main classes will be explained.
	
	\subsection{Node}
	This is the main class, which represent a common group member (also called \textit{participant}). It defines the main functionalities of a group member which communicates under the Virtual Synchrony. These are:
	\begin{itemize}
		\item \textbf{receipt of a message} - when a data message is received, it is delivered to the application, but a copy of it is kept until it becomes \textit{``stable''}, which means the sender was able to send that message to everybody in the group.
		\item \textbf{receipt of a stable message} - upon this message is received, the corresponding copy of the data message can be safely deleted since it is confirmed that the message has been delivered to everybody in the group.
		\item \textbf{view change mechanism} - it is based on the steps described by Virtual Synchrony. For implementation details refer to section \ref{sec:imp_details}.
%		when a view change message is received, each node is responsible to check for any unstable messages. Then, if it is the case, all the unstable message are sent to all the group participants through what is called an \textit{all-to-all} exchange. Finally, each node sends a flush message to signal its intention of changing the view. Every time a flush message is received, it checks if all the flush for that view has been received. If so, it performs a view installation (progress to next view). If there were previous views that were still open, it first install them \textbf{[? , deliver all messages not delivered for that particular]} and then it install the latest \textbf{[? possible]} one.
	\end{itemize}
	Moreover, each node exploits a View object to keep track of which messages have been delivered, which are still unstable and to recognize whether all the flush messages have been received.
	
	\subsection{Participant}
	It represents a member of the group which continuously sends and receives data messages to/from others. It extends the \textit{Node} class providing more functionalities such as send of data messages and the mechanisms to set it as crashed. In addition, it provides a finer implementation of some functionalities as flush messages management and view change.
	
	It inherits all the attributes of \textit{Node} (e.g. the list of received messages, the unstable ones (with respect to each view) and the queue for messages received too early). It also adds some new properties to control the flow and the behaviour of the actors. Other than trivial ones (messageID, MIN\_DELAY, MAX\_DELAY for handling the transmission of new messages) there are some status variables that enforce specific behaviours according to the protocol:
	\begin{itemize}
		\item [-] \textit{\textbf{justEntered}}: it is used just once to trigger a new \textit{SendDatamessage} that makes the process send a new multicast, and reissues itself with a random delay $d \in [\text{MIN\_DELAY, MAX\_DELAY}]$
		\item [-] \textit{\textbf{allowSending}}: as its name tells, when its value is \textbf{false}, the process will not send any new message to the others in the system
		\item [-] \textit{\textbf{crashed}}: tells whether the process has failed or not (its objective is to simulate the crash of a given node); this variable can be set in different manners 
	\end{itemize}
	As described in the Virtual Synchrony protocol, once a node receives a message that notifies a view change, all the unstable messages for the previous epoch are sent to all the participants in the new view, ensuring that all the operational processes receive all the messages sent from everybody within its epoch. This avoids the situation where a node does not receive certain messages from a crashed one, catching up thanks to the non-faulty participants still running.
	This kind of message is differentiated from the \textit{standard} one, since when this kind of messages is received, no copy of it occurs because it can be seen as \textit{already stable}.
	
	\subsection{Group Manager}
	It is a \textit{reliable} group participant responsible to coordinate view updates between group members. It is built on top of the \textit{Node} class and therefore it share most of the characteristics of a group participant. Indeed, it also implements Virtual Syncrony logics, such as reaction to View Change and flush messages, although it is only able to receive messages. Besides, it is in charge of detecting crashes, which are recognized through timeout mechanism based on:
	
	\begin{itemize}
		\item \textbf{missing heartbeat message} - it is a message sent continuously by each cohort to the Group Manager to notify that it is still operational. Group manager detects that a participant has crashed when its heartbeat is not received within a predefined timeout
		\item \textbf{missing flush message} - it is a message sent by all the group members when a view change is triggered. Group Manager detects if a participant is crashed when it does not receive its flush message within the expected predefined timeout
	\end{itemize}
	In both cases, the group manager will trigger a view change, creating a new view and sending it to every group member through multicast exchange.
	
	Apart from previous tasks, Group manager also manage requests from new participants to join the group. To respect the Virtual Synchrony, a new view is generated to keep track of new member acquisition.
	
%	\textbf{We need to specify that participant multicast to everybody but itself, but the checker doesn't take into account that. I mean, the checker doesn't take into account the fact that a node can crash silently. So it signals that there are some messages that are not delivered to all the participant in that epoc, but those who're missing are not operational anymore, so it should be fine.}
	
	\section{Implementation Details}
	\label{sec:imp_details}
	\subsection{Management of flushes and view-changes}
	The implementation in our project is slightly different from the one later proposed by Timofei Istomin. The following pseudo-code and the comment should clarify the process.
	
	\begin{algorithm}
	\begin{algorithmic}
		\REQUIRE $\text{flush message }f_{a}(v_{i}) \text{ from node a for new view } v_{i} $
		\IF {$ \text{all nodes in } v_{i-1} \text{ sent the flush for } v_{i} $}
			\FORALL{$v_{k} \text{ with } i < k \le j $}
				\STATE $ \text{install } v_{k} $
				\STATE $ \text{deliver all messages } m \text{ that were sent in } v_{k} $
			\ENDFOR
			\RETURN $ \exists v_{l}, l > i, s.t. \text{ is still missing flushes}$
		\ENDIF
		\RETURN false
	\end{algorithmic}	
	\end{algorithm}	
	
	When a flush message is received, the process checks whether all the flushes have been collected from the other participants for the installation of a successive view $v_{j}$. If so, all the successive (and pending) views to the current $v_{i}$ are installed in order. This can be the case since faulty nodes may not send the flush message and leave some views pending, thus these ones need to be cascadingly installed once a more recent one meets the requisites.\newline
	Each view $v_{k} \text{ s.t. } i < k \le j$ gets installed and messages queued waiting for the right view are delivered.\newline
	
	The boolean returned valued is then exploited to check if a new view was installed and no other view change are still pending. Only in that case \texttt{allowSending} can be set back to \texttt{true} assuring the desired behaviour.
	Starting sending while the view change process is still running can lead to delivery of messages in wrong epochs from other processes, since they would infer the wrong view from which the message has been sent (and they could not do else ways).
		
	\subsection{Heartbeat}
	Each participant constantly sends a \textit{HeartbeatMessage} to the Group Manager: its function is just to ``prove" the liveness of the process. This message does not get interrupted nor interferes with the behaviour and evolution of the system and it is just a trick to detect silent crashes of the nodes.
	
	\section{Crashes and Joining}
	The system is able to tolerate silent cohort crashes by means of the flush protocol. To control cohorts state, it is possible to send them special messages that make them crash according to predefined behaviour. Crashing modalities are specified below:
	
	\begin{itemize}
		\item \textbf{crash at any time} - immediately set the chosen node as crashed
		\item \textbf{crash on sending} - set the node crashed when it will perform next multicast to the group. As a result, only half of the group members will receive the sent message. It is up to the \textit{all-to-all} exchange, performed during view change, to make all the other operational nodes delivering the sent message.
		\item \textbf{crash on receiving} - set the node crashed when it will receive next message. In this way the node is not able to perform the delivery of received message before crashing.
		\item \textbf{crash on view change} - set the node crashed when next view change is triggered. Therefore, the chosen node will not be able to perform the all-to-all message exchange and to send flush messages.
	\end{itemize}
	
	The system also allows to add further members. To join a new participant, a request must be sent to the group manager, which will fulfill it. Then the Group Manager will generate a new \textit{id} for the new member and it will introduce the participant into the group, triggering a view change.
		
\end{document}
