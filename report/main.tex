\documentclass[11pt]{article}

\usepackage{geometry}
\geometry{
	a4paper,
	top=2.5cm
}
\usepackage[utf8x]{inputenc}
\usepackage{graphicx}
\usepackage{float}
\usepackage[hidelinks]{hyperref}
\usepackage{cleveref}
\usepackage{multirow}
\usepackage{multicol}
\usepackage{amsmath}
\usepackage{enumerate}
\usepackage{algorithm}
\usepackage{algorithmicx}
\usepackage{algpseudocode}
\usepackage{url}

\title{\LARGE{\textbf{Distributed Systems 1 Project Report\\\small{Virtual Synchrony}}}}
\author{Daniele Bissoli - 197810, Andrea Zampieri - 198762}
\date{}

\begin{document}
	\maketitle
	
	\section{Participant Class}
		The so called \textit{Participant} is the implementation of an active node in the system: it sends messages to the other nodes and delivers the received ones according to the Virtual Synchrony logic.\newline
		It inherits all the attributes of \textit{Node} (e.g. the list of received messages, the unstable ones (with respect to each view) and the queue for messages received too early), and adds some of its own in order to be able to control the flow and the behaviour of the actors. Other than trivial ones (messageID, MIN\_DELAY, MAX\_DELAY for handling the transmission of new messages) there are some status variables to enforce the wanted behaviour. 
		\begin{itemize}
			\item [-] \textit{\textbf{justEntered}}: it's used just once to trigger a new \textit{SendDatamessage} that makes the process send a new multicast, and reissues itself with a random delay $d \in [\text{MIN\_DELAY,MAX\_DELAY}]$
			\item [-] \textit{\textbf{allowSending}}: as its name tells, when its value is \textbf{false}, the process won't send any new message to the others in the system
			\item [-] \textit{\textbf{crashed}}: tells wether the process has failed or not (it's used to simulate the crash of a given process)
		\end{itemize}
		
\end{document}